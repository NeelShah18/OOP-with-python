
% Default to the notebook output style

    


% Inherit from the specified cell style.




    
\documentclass[11pt]{article}

    
    
    \usepackage[T1]{fontenc}
    % Nicer default font (+ math font) than Computer Modern for most use cases
    \usepackage{mathpazo}

    % Basic figure setup, for now with no caption control since it's done
    % automatically by Pandoc (which extracts ![](path) syntax from Markdown).
    \usepackage{graphicx}
    % We will generate all images so they have a width \maxwidth. This means
    % that they will get their normal width if they fit onto the page, but
    % are scaled down if they would overflow the margins.
    \makeatletter
    \def\maxwidth{\ifdim\Gin@nat@width>\linewidth\linewidth
    \else\Gin@nat@width\fi}
    \makeatother
    \let\Oldincludegraphics\includegraphics
    % Set max figure width to be 80% of text width, for now hardcoded.
    \renewcommand{\includegraphics}[1]{\Oldincludegraphics[width=.8\maxwidth]{#1}}
    % Ensure that by default, figures have no caption (until we provide a
    % proper Figure object with a Caption API and a way to capture that
    % in the conversion process - todo).
    \usepackage{caption}
    \DeclareCaptionLabelFormat{nolabel}{}
    \captionsetup{labelformat=nolabel}

    \usepackage{adjustbox} % Used to constrain images to a maximum size 
    \usepackage{xcolor} % Allow colors to be defined
    \usepackage{enumerate} % Needed for markdown enumerations to work
    \usepackage{geometry} % Used to adjust the document margins
    \usepackage{amsmath} % Equations
    \usepackage{amssymb} % Equations
    \usepackage{textcomp} % defines textquotesingle
    % Hack from http://tex.stackexchange.com/a/47451/13684:
    \AtBeginDocument{%
        \def\PYZsq{\textquotesingle}% Upright quotes in Pygmentized code
    }
    \usepackage{upquote} % Upright quotes for verbatim code
    \usepackage{eurosym} % defines \euro
    \usepackage[mathletters]{ucs} % Extended unicode (utf-8) support
    \usepackage[utf8x]{inputenc} % Allow utf-8 characters in the tex document
    \usepackage{fancyvrb} % verbatim replacement that allows latex
    \usepackage{grffile} % extends the file name processing of package graphics 
                         % to support a larger range 
    % The hyperref package gives us a pdf with properly built
    % internal navigation ('pdf bookmarks' for the table of contents,
    % internal cross-reference links, web links for URLs, etc.)
    \usepackage{hyperref}
    \usepackage{longtable} % longtable support required by pandoc >1.10
    \usepackage{booktabs}  % table support for pandoc > 1.12.2
    \usepackage[inline]{enumitem} % IRkernel/repr support (it uses the enumerate* environment)
    \usepackage[normalem]{ulem} % ulem is needed to support strikethroughs (\sout)
                                % normalem makes italics be italics, not underlines
    

    
    
    % Colors for the hyperref package
    \definecolor{urlcolor}{rgb}{0,.145,.698}
    \definecolor{linkcolor}{rgb}{.71,0.21,0.01}
    \definecolor{citecolor}{rgb}{.12,.54,.11}

    % ANSI colors
    \definecolor{ansi-black}{HTML}{3E424D}
    \definecolor{ansi-black-intense}{HTML}{282C36}
    \definecolor{ansi-red}{HTML}{E75C58}
    \definecolor{ansi-red-intense}{HTML}{B22B31}
    \definecolor{ansi-green}{HTML}{00A250}
    \definecolor{ansi-green-intense}{HTML}{007427}
    \definecolor{ansi-yellow}{HTML}{DDB62B}
    \definecolor{ansi-yellow-intense}{HTML}{B27D12}
    \definecolor{ansi-blue}{HTML}{208FFB}
    \definecolor{ansi-blue-intense}{HTML}{0065CA}
    \definecolor{ansi-magenta}{HTML}{D160C4}
    \definecolor{ansi-magenta-intense}{HTML}{A03196}
    \definecolor{ansi-cyan}{HTML}{60C6C8}
    \definecolor{ansi-cyan-intense}{HTML}{258F8F}
    \definecolor{ansi-white}{HTML}{C5C1B4}
    \definecolor{ansi-white-intense}{HTML}{A1A6B2}

    % commands and environments needed by pandoc snippets
    % extracted from the output of `pandoc -s`
    \providecommand{\tightlist}{%
      \setlength{\itemsep}{0pt}\setlength{\parskip}{0pt}}
    \DefineVerbatimEnvironment{Highlighting}{Verbatim}{commandchars=\\\{\}}
    % Add ',fontsize=\small' for more characters per line
    \newenvironment{Shaded}{}{}
    \newcommand{\KeywordTok}[1]{\textcolor[rgb]{0.00,0.44,0.13}{\textbf{{#1}}}}
    \newcommand{\DataTypeTok}[1]{\textcolor[rgb]{0.56,0.13,0.00}{{#1}}}
    \newcommand{\DecValTok}[1]{\textcolor[rgb]{0.25,0.63,0.44}{{#1}}}
    \newcommand{\BaseNTok}[1]{\textcolor[rgb]{0.25,0.63,0.44}{{#1}}}
    \newcommand{\FloatTok}[1]{\textcolor[rgb]{0.25,0.63,0.44}{{#1}}}
    \newcommand{\CharTok}[1]{\textcolor[rgb]{0.25,0.44,0.63}{{#1}}}
    \newcommand{\StringTok}[1]{\textcolor[rgb]{0.25,0.44,0.63}{{#1}}}
    \newcommand{\CommentTok}[1]{\textcolor[rgb]{0.38,0.63,0.69}{\textit{{#1}}}}
    \newcommand{\OtherTok}[1]{\textcolor[rgb]{0.00,0.44,0.13}{{#1}}}
    \newcommand{\AlertTok}[1]{\textcolor[rgb]{1.00,0.00,0.00}{\textbf{{#1}}}}
    \newcommand{\FunctionTok}[1]{\textcolor[rgb]{0.02,0.16,0.49}{{#1}}}
    \newcommand{\RegionMarkerTok}[1]{{#1}}
    \newcommand{\ErrorTok}[1]{\textcolor[rgb]{1.00,0.00,0.00}{\textbf{{#1}}}}
    \newcommand{\NormalTok}[1]{{#1}}
    
    % Additional commands for more recent versions of Pandoc
    \newcommand{\ConstantTok}[1]{\textcolor[rgb]{0.53,0.00,0.00}{{#1}}}
    \newcommand{\SpecialCharTok}[1]{\textcolor[rgb]{0.25,0.44,0.63}{{#1}}}
    \newcommand{\VerbatimStringTok}[1]{\textcolor[rgb]{0.25,0.44,0.63}{{#1}}}
    \newcommand{\SpecialStringTok}[1]{\textcolor[rgb]{0.73,0.40,0.53}{{#1}}}
    \newcommand{\ImportTok}[1]{{#1}}
    \newcommand{\DocumentationTok}[1]{\textcolor[rgb]{0.73,0.13,0.13}{\textit{{#1}}}}
    \newcommand{\AnnotationTok}[1]{\textcolor[rgb]{0.38,0.63,0.69}{\textbf{\textit{{#1}}}}}
    \newcommand{\CommentVarTok}[1]{\textcolor[rgb]{0.38,0.63,0.69}{\textbf{\textit{{#1}}}}}
    \newcommand{\VariableTok}[1]{\textcolor[rgb]{0.10,0.09,0.49}{{#1}}}
    \newcommand{\ControlFlowTok}[1]{\textcolor[rgb]{0.00,0.44,0.13}{\textbf{{#1}}}}
    \newcommand{\OperatorTok}[1]{\textcolor[rgb]{0.40,0.40,0.40}{{#1}}}
    \newcommand{\BuiltInTok}[1]{{#1}}
    \newcommand{\ExtensionTok}[1]{{#1}}
    \newcommand{\PreprocessorTok}[1]{\textcolor[rgb]{0.74,0.48,0.00}{{#1}}}
    \newcommand{\AttributeTok}[1]{\textcolor[rgb]{0.49,0.56,0.16}{{#1}}}
    \newcommand{\InformationTok}[1]{\textcolor[rgb]{0.38,0.63,0.69}{\textbf{\textit{{#1}}}}}
    \newcommand{\WarningTok}[1]{\textcolor[rgb]{0.38,0.63,0.69}{\textbf{\textit{{#1}}}}}
    
    
    % Define a nice break command that doesn't care if a line doesn't already
    % exist.
    \def\br{\hspace*{\fill} \\* }
    % Math Jax compatability definitions
    \def\gt{>}
    \def\lt{<}
    % Document parameters
    \title{OOP with python}
    
    
    

    % Pygments definitions
    
\makeatletter
\def\PY@reset{\let\PY@it=\relax \let\PY@bf=\relax%
    \let\PY@ul=\relax \let\PY@tc=\relax%
    \let\PY@bc=\relax \let\PY@ff=\relax}
\def\PY@tok#1{\csname PY@tok@#1\endcsname}
\def\PY@toks#1+{\ifx\relax#1\empty\else%
    \PY@tok{#1}\expandafter\PY@toks\fi}
\def\PY@do#1{\PY@bc{\PY@tc{\PY@ul{%
    \PY@it{\PY@bf{\PY@ff{#1}}}}}}}
\def\PY#1#2{\PY@reset\PY@toks#1+\relax+\PY@do{#2}}

\expandafter\def\csname PY@tok@w\endcsname{\def\PY@tc##1{\textcolor[rgb]{0.73,0.73,0.73}{##1}}}
\expandafter\def\csname PY@tok@c\endcsname{\let\PY@it=\textit\def\PY@tc##1{\textcolor[rgb]{0.25,0.50,0.50}{##1}}}
\expandafter\def\csname PY@tok@cp\endcsname{\def\PY@tc##1{\textcolor[rgb]{0.74,0.48,0.00}{##1}}}
\expandafter\def\csname PY@tok@k\endcsname{\let\PY@bf=\textbf\def\PY@tc##1{\textcolor[rgb]{0.00,0.50,0.00}{##1}}}
\expandafter\def\csname PY@tok@kp\endcsname{\def\PY@tc##1{\textcolor[rgb]{0.00,0.50,0.00}{##1}}}
\expandafter\def\csname PY@tok@kt\endcsname{\def\PY@tc##1{\textcolor[rgb]{0.69,0.00,0.25}{##1}}}
\expandafter\def\csname PY@tok@o\endcsname{\def\PY@tc##1{\textcolor[rgb]{0.40,0.40,0.40}{##1}}}
\expandafter\def\csname PY@tok@ow\endcsname{\let\PY@bf=\textbf\def\PY@tc##1{\textcolor[rgb]{0.67,0.13,1.00}{##1}}}
\expandafter\def\csname PY@tok@nb\endcsname{\def\PY@tc##1{\textcolor[rgb]{0.00,0.50,0.00}{##1}}}
\expandafter\def\csname PY@tok@nf\endcsname{\def\PY@tc##1{\textcolor[rgb]{0.00,0.00,1.00}{##1}}}
\expandafter\def\csname PY@tok@nc\endcsname{\let\PY@bf=\textbf\def\PY@tc##1{\textcolor[rgb]{0.00,0.00,1.00}{##1}}}
\expandafter\def\csname PY@tok@nn\endcsname{\let\PY@bf=\textbf\def\PY@tc##1{\textcolor[rgb]{0.00,0.00,1.00}{##1}}}
\expandafter\def\csname PY@tok@ne\endcsname{\let\PY@bf=\textbf\def\PY@tc##1{\textcolor[rgb]{0.82,0.25,0.23}{##1}}}
\expandafter\def\csname PY@tok@nv\endcsname{\def\PY@tc##1{\textcolor[rgb]{0.10,0.09,0.49}{##1}}}
\expandafter\def\csname PY@tok@no\endcsname{\def\PY@tc##1{\textcolor[rgb]{0.53,0.00,0.00}{##1}}}
\expandafter\def\csname PY@tok@nl\endcsname{\def\PY@tc##1{\textcolor[rgb]{0.63,0.63,0.00}{##1}}}
\expandafter\def\csname PY@tok@ni\endcsname{\let\PY@bf=\textbf\def\PY@tc##1{\textcolor[rgb]{0.60,0.60,0.60}{##1}}}
\expandafter\def\csname PY@tok@na\endcsname{\def\PY@tc##1{\textcolor[rgb]{0.49,0.56,0.16}{##1}}}
\expandafter\def\csname PY@tok@nt\endcsname{\let\PY@bf=\textbf\def\PY@tc##1{\textcolor[rgb]{0.00,0.50,0.00}{##1}}}
\expandafter\def\csname PY@tok@nd\endcsname{\def\PY@tc##1{\textcolor[rgb]{0.67,0.13,1.00}{##1}}}
\expandafter\def\csname PY@tok@s\endcsname{\def\PY@tc##1{\textcolor[rgb]{0.73,0.13,0.13}{##1}}}
\expandafter\def\csname PY@tok@sd\endcsname{\let\PY@it=\textit\def\PY@tc##1{\textcolor[rgb]{0.73,0.13,0.13}{##1}}}
\expandafter\def\csname PY@tok@si\endcsname{\let\PY@bf=\textbf\def\PY@tc##1{\textcolor[rgb]{0.73,0.40,0.53}{##1}}}
\expandafter\def\csname PY@tok@se\endcsname{\let\PY@bf=\textbf\def\PY@tc##1{\textcolor[rgb]{0.73,0.40,0.13}{##1}}}
\expandafter\def\csname PY@tok@sr\endcsname{\def\PY@tc##1{\textcolor[rgb]{0.73,0.40,0.53}{##1}}}
\expandafter\def\csname PY@tok@ss\endcsname{\def\PY@tc##1{\textcolor[rgb]{0.10,0.09,0.49}{##1}}}
\expandafter\def\csname PY@tok@sx\endcsname{\def\PY@tc##1{\textcolor[rgb]{0.00,0.50,0.00}{##1}}}
\expandafter\def\csname PY@tok@m\endcsname{\def\PY@tc##1{\textcolor[rgb]{0.40,0.40,0.40}{##1}}}
\expandafter\def\csname PY@tok@gh\endcsname{\let\PY@bf=\textbf\def\PY@tc##1{\textcolor[rgb]{0.00,0.00,0.50}{##1}}}
\expandafter\def\csname PY@tok@gu\endcsname{\let\PY@bf=\textbf\def\PY@tc##1{\textcolor[rgb]{0.50,0.00,0.50}{##1}}}
\expandafter\def\csname PY@tok@gd\endcsname{\def\PY@tc##1{\textcolor[rgb]{0.63,0.00,0.00}{##1}}}
\expandafter\def\csname PY@tok@gi\endcsname{\def\PY@tc##1{\textcolor[rgb]{0.00,0.63,0.00}{##1}}}
\expandafter\def\csname PY@tok@gr\endcsname{\def\PY@tc##1{\textcolor[rgb]{1.00,0.00,0.00}{##1}}}
\expandafter\def\csname PY@tok@ge\endcsname{\let\PY@it=\textit}
\expandafter\def\csname PY@tok@gs\endcsname{\let\PY@bf=\textbf}
\expandafter\def\csname PY@tok@gp\endcsname{\let\PY@bf=\textbf\def\PY@tc##1{\textcolor[rgb]{0.00,0.00,0.50}{##1}}}
\expandafter\def\csname PY@tok@go\endcsname{\def\PY@tc##1{\textcolor[rgb]{0.53,0.53,0.53}{##1}}}
\expandafter\def\csname PY@tok@gt\endcsname{\def\PY@tc##1{\textcolor[rgb]{0.00,0.27,0.87}{##1}}}
\expandafter\def\csname PY@tok@err\endcsname{\def\PY@bc##1{\setlength{\fboxsep}{0pt}\fcolorbox[rgb]{1.00,0.00,0.00}{1,1,1}{\strut ##1}}}
\expandafter\def\csname PY@tok@kc\endcsname{\let\PY@bf=\textbf\def\PY@tc##1{\textcolor[rgb]{0.00,0.50,0.00}{##1}}}
\expandafter\def\csname PY@tok@kd\endcsname{\let\PY@bf=\textbf\def\PY@tc##1{\textcolor[rgb]{0.00,0.50,0.00}{##1}}}
\expandafter\def\csname PY@tok@kn\endcsname{\let\PY@bf=\textbf\def\PY@tc##1{\textcolor[rgb]{0.00,0.50,0.00}{##1}}}
\expandafter\def\csname PY@tok@kr\endcsname{\let\PY@bf=\textbf\def\PY@tc##1{\textcolor[rgb]{0.00,0.50,0.00}{##1}}}
\expandafter\def\csname PY@tok@bp\endcsname{\def\PY@tc##1{\textcolor[rgb]{0.00,0.50,0.00}{##1}}}
\expandafter\def\csname PY@tok@fm\endcsname{\def\PY@tc##1{\textcolor[rgb]{0.00,0.00,1.00}{##1}}}
\expandafter\def\csname PY@tok@vc\endcsname{\def\PY@tc##1{\textcolor[rgb]{0.10,0.09,0.49}{##1}}}
\expandafter\def\csname PY@tok@vg\endcsname{\def\PY@tc##1{\textcolor[rgb]{0.10,0.09,0.49}{##1}}}
\expandafter\def\csname PY@tok@vi\endcsname{\def\PY@tc##1{\textcolor[rgb]{0.10,0.09,0.49}{##1}}}
\expandafter\def\csname PY@tok@vm\endcsname{\def\PY@tc##1{\textcolor[rgb]{0.10,0.09,0.49}{##1}}}
\expandafter\def\csname PY@tok@sa\endcsname{\def\PY@tc##1{\textcolor[rgb]{0.73,0.13,0.13}{##1}}}
\expandafter\def\csname PY@tok@sb\endcsname{\def\PY@tc##1{\textcolor[rgb]{0.73,0.13,0.13}{##1}}}
\expandafter\def\csname PY@tok@sc\endcsname{\def\PY@tc##1{\textcolor[rgb]{0.73,0.13,0.13}{##1}}}
\expandafter\def\csname PY@tok@dl\endcsname{\def\PY@tc##1{\textcolor[rgb]{0.73,0.13,0.13}{##1}}}
\expandafter\def\csname PY@tok@s2\endcsname{\def\PY@tc##1{\textcolor[rgb]{0.73,0.13,0.13}{##1}}}
\expandafter\def\csname PY@tok@sh\endcsname{\def\PY@tc##1{\textcolor[rgb]{0.73,0.13,0.13}{##1}}}
\expandafter\def\csname PY@tok@s1\endcsname{\def\PY@tc##1{\textcolor[rgb]{0.73,0.13,0.13}{##1}}}
\expandafter\def\csname PY@tok@mb\endcsname{\def\PY@tc##1{\textcolor[rgb]{0.40,0.40,0.40}{##1}}}
\expandafter\def\csname PY@tok@mf\endcsname{\def\PY@tc##1{\textcolor[rgb]{0.40,0.40,0.40}{##1}}}
\expandafter\def\csname PY@tok@mh\endcsname{\def\PY@tc##1{\textcolor[rgb]{0.40,0.40,0.40}{##1}}}
\expandafter\def\csname PY@tok@mi\endcsname{\def\PY@tc##1{\textcolor[rgb]{0.40,0.40,0.40}{##1}}}
\expandafter\def\csname PY@tok@il\endcsname{\def\PY@tc##1{\textcolor[rgb]{0.40,0.40,0.40}{##1}}}
\expandafter\def\csname PY@tok@mo\endcsname{\def\PY@tc##1{\textcolor[rgb]{0.40,0.40,0.40}{##1}}}
\expandafter\def\csname PY@tok@ch\endcsname{\let\PY@it=\textit\def\PY@tc##1{\textcolor[rgb]{0.25,0.50,0.50}{##1}}}
\expandafter\def\csname PY@tok@cm\endcsname{\let\PY@it=\textit\def\PY@tc##1{\textcolor[rgb]{0.25,0.50,0.50}{##1}}}
\expandafter\def\csname PY@tok@cpf\endcsname{\let\PY@it=\textit\def\PY@tc##1{\textcolor[rgb]{0.25,0.50,0.50}{##1}}}
\expandafter\def\csname PY@tok@c1\endcsname{\let\PY@it=\textit\def\PY@tc##1{\textcolor[rgb]{0.25,0.50,0.50}{##1}}}
\expandafter\def\csname PY@tok@cs\endcsname{\let\PY@it=\textit\def\PY@tc##1{\textcolor[rgb]{0.25,0.50,0.50}{##1}}}

\def\PYZbs{\char`\\}
\def\PYZus{\char`\_}
\def\PYZob{\char`\{}
\def\PYZcb{\char`\}}
\def\PYZca{\char`\^}
\def\PYZam{\char`\&}
\def\PYZlt{\char`\<}
\def\PYZgt{\char`\>}
\def\PYZsh{\char`\#}
\def\PYZpc{\char`\%}
\def\PYZdl{\char`\$}
\def\PYZhy{\char`\-}
\def\PYZsq{\char`\'}
\def\PYZdq{\char`\"}
\def\PYZti{\char`\~}
% for compatibility with earlier versions
\def\PYZat{@}
\def\PYZlb{[}
\def\PYZrb{]}
\makeatother


    % Exact colors from NB
    \definecolor{incolor}{rgb}{0.0, 0.0, 0.5}
    \definecolor{outcolor}{rgb}{0.545, 0.0, 0.0}



    
    % Prevent overflowing lines due to hard-to-break entities
    \sloppy 
    % Setup hyperref package
    \hypersetup{
      breaklinks=true,  % so long urls are correctly broken across lines
      colorlinks=true,
      urlcolor=urlcolor,
      linkcolor=linkcolor,
      citecolor=citecolor,
      }
    % Slightly bigger margins than the latex defaults
    
    \geometry{verbose,tmargin=1in,bmargin=1in,lmargin=1in,rmargin=1in}
    
    

    \begin{document}
    
    
    \maketitle
    
    

    
    \section{Hi, I am Neel Shah}\label{hi-i-am-neel-shah}

\subsection{Wroking at Datalog.ai and IDLI on Data analyst, Machine
learnig and Deep
Learning}\label{wroking-at-datalog.ai-and-idli-on-data-analyst-machine-learnig-and-deep-learning}

Contact details:

\begin{enumerate}
\def\labelenumi{\arabic{enumi})}
\item
  Website: https://neelshah18.github.io/
\item
  GitHub: https://github.com/NeelShah18
\item
  LinkedIn: https://www.linkedin.com/in/neel-shah-7b5495104/
\item
  Facebook: https://www.facebook.com/neelxyz
\end{enumerate}

*** all code and data are available on my GitHub account under MIT open
source licence***

    \begin{itemize}
\tightlist
\item
  Though many computer scientists and programmers consider OOP to be a
  modern programming paradigm, the roots go back to 1960s. The first
  programming language to use objects was Simula 67. As the name
  implies, Simula 67 was introduced in the year 1967. A major
  breakthrough for object-oriented programming came with the programming
  language Smalltalk in the 1970s.
\end{itemize}

\paragraph{Following two concept is explain everything the use of
OOPs.}\label{following-two-concept-is-explain-everything-the-use-of-oops.}

\begin{enumerate}
\def\labelenumi{\arabic{enumi}.}
\item
  Duplicate code is a Bad.
\item
  Code will always be changed.
\end{enumerate}

\paragraph{Object-Oriented Programming has the following
advantages:}\label{object-oriented-programming-has-the-following-advantages}

\begin{itemize}
\tightlist
\item
  OOP provides a clear modular structure for programs which makes it
  good for defining abstract datatypes where implementation details are
  hidden and the unit has a clearly defined interface.
\item
  OOP makes it easy to maintain and modify existing code as new objects
  can be created with small differences to existing ones.
\item
  OOP provides a good framework for code libraries where supplied
  software components can be easily adapted and modified by the
  programmer. This is particularly useful for developing graphical user
  interfaces.
\end{itemize}

    \paragraph{Terminology:}\label{terminology}

\begin{itemize}
\item
  \textbf{Class}: A user-defined prototype for an object that defines a
  set of attributes that characterize any object of the class. The
  attributes are data members (class variables and instance variables)
  and methods, accessed via dot notation.
\item
  \textbf{Class variable}: A variable that is shared by all instances of
  a class. Class variables are defined within a class but outside any of
  the class's methods. Class variables are not used as frequently as
  instance variables are.
\item
  \textbf{Data member}: A class variable or instance variable that holds
  data associated with a class and its objects.
\item
  \textbf{Instance variable}: A variable that is defined inside a method
  and belongs only to the current instance of a class.
\item
  \textbf{Inheritance}: The transfer of the characteristics of a class
  to other classes that are derived from it.
\item
  \textbf{Instance}: An individual object of a certain class. An object
  obj that belongs to a class Circle, for example, is an instance of the
  class Circle.
\item
  \textbf{Method}: A special kind of function that is defined in a class
  definition.
\item
  \textbf{Object}: A unique instance of a data structure that's defined
  by its class. An object comprises both data members (class variables
  and instance variables) and methods.
\item
  \textbf{Function overloading}: The assignment of more than one
  behavior to a particular function. The operation performed varies by
  the types of objects or arguments involved.
\end{itemize}

    \paragraph{Everything in class is
python:}\label{everything-in-class-is-python}

Even though we haven't talked about classes and object orientation in
previous chapters, we have worked with classes all the time. In fact,
everything is a class in Python.

    \begin{Verbatim}[commandchars=\\\{\}]
{\color{incolor}In [{\color{incolor}5}]:} \PY{k+kn}{import} \PY{n+nn}{math}
        
        \PY{n}{x} \PY{o}{=} \PY{l+m+mi}{4}
        \PY{n+nb}{print}\PY{p}{(}\PY{n+nb}{type}\PY{p}{(}\PY{n}{x}\PY{p}{)}\PY{p}{)}
        
        \PY{k}{def} \PY{n+nf}{f}\PY{p}{(}\PY{n}{x}\PY{p}{)}\PY{p}{:}
            \PY{k}{return} \PY{n}{x}\PY{o}{+}\PY{l+m+mi}{1}
        \PY{n+nb}{print}\PY{p}{(}\PY{n+nb}{type}\PY{p}{(}\PY{n}{f}\PY{p}{)}\PY{p}{)}
        \PY{n+nb}{print}\PY{p}{(}\PY{n+nb}{type}\PY{p}{(}\PY{n}{math}\PY{p}{)}\PY{p}{)}
        \PY{l+s+sd}{\PYZsq{}\PYZsq{}\PYZsq{}As you can see, everything in python is class.\PYZsq{}\PYZsq{}\PYZsq{}}
\end{Verbatim}

    \begin{Verbatim}[commandchars=\\\{\}]
<class 'int'>
<class 'function'>
<class 'module'>

    \end{Verbatim}

            \begin{Verbatim}[commandchars=\\\{\}]
{\color{outcolor}Out[{\color{outcolor}5}]:} 'As you can see, everything in python is class.'
\end{Verbatim}
        
    \begin{Verbatim}[commandchars=\\\{\}]
{\color{incolor}In [{\color{incolor}6}]:} \PY{c+c1}{\PYZsh{} Creating Classes}
        
        \PY{k}{class} \PY{n+nc}{Robot}\PY{p}{:}
            \PY{k}{pass}
        
        \PY{k}{if} \PY{n+nv+vm}{\PYZus{}\PYZus{}name\PYZus{}\PYZus{}} \PY{o}{==} \PY{l+s+s2}{\PYZdq{}}\PY{l+s+s2}{\PYZus{}\PYZus{}main\PYZus{}\PYZus{}}\PY{l+s+s2}{\PYZdq{}}\PY{p}{:}
            \PY{n}{x} \PY{o}{=} \PY{n}{Robot}\PY{p}{(}\PY{p}{)}
            \PY{n}{y} \PY{o}{=} \PY{n}{Robot}\PY{p}{(}\PY{p}{)}
            \PY{n}{y2} \PY{o}{=} \PY{n}{y}
            \PY{n+nb}{print}\PY{p}{(}\PY{n}{y} \PY{o}{==} \PY{n}{y2}\PY{p}{)}
            \PY{n+nb}{print}\PY{p}{(}\PY{n}{y} \PY{o}{==} \PY{n}{x}\PY{p}{)}
\end{Verbatim}

    \begin{Verbatim}[commandchars=\\\{\}]
True
False

    \end{Verbatim}

    \begin{Verbatim}[commandchars=\\\{\}]
{\color{incolor}In [{\color{incolor}19}]:} \PY{k}{class} \PY{n+nc}{Customer}\PY{p}{(}\PY{n+nb}{object}\PY{p}{)}\PY{p}{:}
             \PY{l+s+sd}{\PYZdq{}\PYZdq{}\PYZdq{}A customer of ABC Bank with a checking account. Customers have the}
         \PY{l+s+sd}{    following properties:}
         
         \PY{l+s+sd}{    Attributes:}
         \PY{l+s+sd}{        name: A string representing the customer\PYZsq{}s name.}
         \PY{l+s+sd}{        balance: A float tracking the current balance of the customer\PYZsq{}s account.}
         \PY{l+s+sd}{    \PYZdq{}\PYZdq{}\PYZdq{}}
         
             \PY{k}{def} \PY{n+nf}{\PYZus{}\PYZus{}init\PYZus{}\PYZus{}}\PY{p}{(}\PY{n+nb+bp}{self}\PY{p}{,} \PY{n}{name}\PY{p}{,} \PY{n}{balance}\PY{o}{=}\PY{l+m+mf}{0.0}\PY{p}{)}\PY{p}{:}
                 \PY{l+s+sd}{\PYZdq{}\PYZdq{}\PYZdq{}Return a Customer object whose name is *name* and starting}
         \PY{l+s+sd}{        balance is *balance*.\PYZdq{}\PYZdq{}\PYZdq{}}
                 \PY{n+nb+bp}{self}\PY{o}{.}\PY{n}{name} \PY{o}{=} \PY{n}{name}
                 \PY{n+nb+bp}{self}\PY{o}{.}\PY{n}{balance} \PY{o}{=} \PY{n}{balance}
         
             \PY{k}{def} \PY{n+nf}{withdraw}\PY{p}{(}\PY{n+nb+bp}{self}\PY{p}{,} \PY{n}{amount}\PY{p}{)}\PY{p}{:}
                 \PY{l+s+sd}{\PYZdq{}\PYZdq{}\PYZdq{}Return the balance remaining after withdrawing *amount*}
         \PY{l+s+sd}{        dollars.\PYZdq{}\PYZdq{}\PYZdq{}}
                 \PY{k}{if} \PY{n}{amount} \PY{o}{\PYZgt{}} \PY{n+nb+bp}{self}\PY{o}{.}\PY{n}{balance}\PY{p}{:}
                     \PY{k}{raise} \PY{n+ne}{RuntimeError}\PY{p}{(}\PY{l+s+s1}{\PYZsq{}}\PY{l+s+s1}{Amount greater than available balance.}\PY{l+s+s1}{\PYZsq{}}\PY{p}{)}
                 \PY{n+nb+bp}{self}\PY{o}{.}\PY{n}{balance} \PY{o}{\PYZhy{}}\PY{o}{=} \PY{n}{amount}
                 \PY{k}{return} \PY{n+nb+bp}{self}\PY{o}{.}\PY{n}{balance}
         
             \PY{k}{def} \PY{n+nf}{deposit}\PY{p}{(}\PY{n+nb+bp}{self}\PY{p}{,} \PY{n}{amount}\PY{p}{)}\PY{p}{:}
                 \PY{l+s+sd}{\PYZdq{}\PYZdq{}\PYZdq{}Return the balance remaining after depositing *amount*}
         \PY{l+s+sd}{        dollars.\PYZdq{}\PYZdq{}\PYZdq{}}
                 \PY{n+nb+bp}{self}\PY{o}{.}\PY{n}{balance} \PY{o}{+}\PY{o}{=} \PY{n}{amount}
                 \PY{k}{return} \PY{n+nb+bp}{self}\PY{o}{.}\PY{n}{balance}
         
         \PY{n}{neel} \PY{o}{=} \PY{n}{Customer}\PY{p}{(}\PY{n}{neel}\PY{p}{,} \PY{l+m+mi}{500}\PY{p}{)}
         \PY{n+nb}{print}\PY{p}{(}\PY{n}{neel}\PY{o}{.}\PY{n}{deposit}\PY{p}{(}\PY{l+m+mi}{200}\PY{p}{)}\PY{p}{)}
         \PY{n+nb}{print}\PY{p}{(}\PY{n}{neel}\PY{o}{.}\PY{n}{withdraw}\PY{p}{(}\PY{l+m+mi}{150}\PY{p}{)}\PY{p}{)}
         
         \PY{l+s+sd}{\PYZsq{}\PYZsq{}\PYZsq{}This is basic example of class in python with \PYZus{}\PYZus{}init\PYZus{}\PYZus{}, methods\PYZsq{}\PYZsq{}\PYZsq{}}
\end{Verbatim}

    \begin{Verbatim}[commandchars=\\\{\}]
700
550

    \end{Verbatim}

            \begin{Verbatim}[commandchars=\\\{\}]
{\color{outcolor}Out[{\color{outcolor}19}]:} 'This is basic example of class in python with \_\_init\_\_, methods'
\end{Verbatim}
        
    The first method \textbf{init}() is a special method, which is called
class constructor or initialization method that Python calls when you
create a new instance of this class.

You declare other class methods like normal functions with the exception
that the first argument to each method is self. Python adds the self
argument to the list for you; you do not need to include it when you
call the methods.

So what's with that self parameter to all of the Customer methods? What
is it? Why, it's the instance, of course! Put another way, a method like
withdraw defines the instructions for withdrawing money from some
abstract customer's account. Calling jeff.withdraw(100.0) puts those
instructions to use on the jeff instance.

    \begin{Verbatim}[commandchars=\\\{\}]
{\color{incolor}In [{\color{incolor}8}]:} \PY{c+c1}{\PYZsh{} one more example}
        
        \PY{k}{class} \PY{n+nc}{Robot}\PY{p}{:}
         
            \PY{k}{def} \PY{n+nf}{\PYZus{}\PYZus{}init\PYZus{}\PYZus{}}\PY{p}{(}\PY{n+nb+bp}{self}\PY{p}{,} \PY{n}{name}\PY{o}{=}\PY{k+kc}{None}\PY{p}{)}\PY{p}{:}
                \PY{n+nb+bp}{self}\PY{o}{.}\PY{n}{name} \PY{o}{=} \PY{n}{name}   
                
            \PY{k}{def} \PY{n+nf}{say\PYZus{}hi}\PY{p}{(}\PY{n+nb+bp}{self}\PY{p}{)}\PY{p}{:}
                \PY{k}{if} \PY{n+nb+bp}{self}\PY{o}{.}\PY{n}{name}\PY{p}{:}
                    \PY{n+nb}{print}\PY{p}{(}\PY{l+s+s2}{\PYZdq{}}\PY{l+s+s2}{Hi, I am }\PY{l+s+s2}{\PYZdq{}} \PY{o}{+} \PY{n+nb+bp}{self}\PY{o}{.}\PY{n}{name}\PY{p}{)}
                \PY{k}{else}\PY{p}{:}
                    \PY{n+nb}{print}\PY{p}{(}\PY{l+s+s2}{\PYZdq{}}\PY{l+s+s2}{Hi, I am a robot without a name}\PY{l+s+s2}{\PYZdq{}}\PY{p}{)}
            
        
        \PY{n}{x} \PY{o}{=} \PY{n}{Robot}\PY{p}{(}\PY{p}{)}
        \PY{n}{x}\PY{o}{.}\PY{n}{say\PYZus{}hi}\PY{p}{(}\PY{p}{)}
        \PY{n}{y} \PY{o}{=} \PY{n}{Robot}\PY{p}{(}\PY{l+s+s2}{\PYZdq{}}\PY{l+s+s2}{Marvin}\PY{l+s+s2}{\PYZdq{}}\PY{p}{)}
        \PY{n}{y}\PY{o}{.}\PY{n}{say\PYZus{}hi}\PY{p}{(}\PY{p}{)}
\end{Verbatim}

    \begin{Verbatim}[commandchars=\\\{\}]
Hi, I am a robot without a name
Hi, I am Marvin

    \end{Verbatim}

    \paragraph{Public- Protected- and Private
Attributes}\label{public--protected--and-private-attributes}

Python uses a special naming scheme for attributes to control the
accessibility of the attributes. So far, we have used attribute names,
which can be freely used inside or outside of a class definition, as we
have seen. This corresponds to public attributes of course.

There are two ways to restrict the access to class attributes:

\begin{itemize}
\item
  First, we can prefix an attribute name with a leading underscore "\_".
  This marks the attribute as protected. It tells users of the class not
  to use this attribute unless, somebody writes a subclass. We will
  learn about inheritance and subclassing in the next chapter of our
  tutorial.
\item
  Second, we can prefix an attribute name with two leading underscores
  "\_\_". The attribute is now inaccessible and invisible from outside.
  It's neither possible to read nor write to those attributes except
  inside of the class definition itself.5
\end{itemize}

\begin{enumerate}
\def\labelenumi{\arabic{enumi})}
\item
  name-\/-\textgreater{} Public-\/-\textgreater{} These attributes can
  be freely used inside or outside of a class definition.
\item
  \_name-\/-\textgreater{} Protected-\/-\textgreater{} Protected
  attributes should not be used outside of the class definition, unless
  inside of a subclass definition.
\item
  \_\_name-\/-\textgreater{} Private-\/-\textgreater{} This kind of
  attribute is inaccessible and invisible. It's neither possible to read
  nor write to those attributes, except inside of the class definition
  itself.
\end{enumerate}

    \begin{Verbatim}[commandchars=\\\{\}]
{\color{incolor}In [{\color{incolor}15}]:} \PY{c+c1}{\PYZsh{} Attribute accessibility example}
         \PY{k}{class} \PY{n+nc}{A}\PY{p}{(}\PY{p}{)}\PY{p}{:}
             
             \PY{k}{def} \PY{n+nf}{\PYZus{}\PYZus{}init\PYZus{}\PYZus{}}\PY{p}{(}\PY{n+nb+bp}{self}\PY{p}{)}\PY{p}{:}
                 \PY{n+nb+bp}{self}\PY{o}{.}\PY{n}{\PYZus{}\PYZus{}priv} \PY{o}{=} \PY{l+s+s2}{\PYZdq{}}\PY{l+s+s2}{I am private}\PY{l+s+s2}{\PYZdq{}}
                 \PY{n+nb+bp}{self}\PY{o}{.}\PY{n}{\PYZus{}prot} \PY{o}{=} \PY{l+s+s2}{\PYZdq{}}\PY{l+s+s2}{I am protected}\PY{l+s+s2}{\PYZdq{}}
                 \PY{n+nb+bp}{self}\PY{o}{.}\PY{n}{pub} \PY{o}{=} \PY{l+s+s2}{\PYZdq{}}\PY{l+s+s2}{I am public}\PY{l+s+s2}{\PYZdq{}}
         
         \PY{n}{x} \PY{o}{=} \PY{n}{A}\PY{p}{(}\PY{p}{)}
         \PY{n+nb}{print}\PY{p}{(}\PY{n}{x}\PY{o}{.}\PY{n}{pub}\PY{p}{)}
         \PY{n+nb}{print}\PY{p}{(}\PY{n}{x}\PY{o}{.}\PY{n}{pub} \PY{o}{+} \PY{l+s+s2}{\PYZdq{}}\PY{l+s+s2}{ and my value can be changed}\PY{l+s+s2}{\PYZdq{}}\PY{p}{)}
         \PY{n+nb}{print}\PY{p}{(}\PY{n}{x}\PY{o}{.}\PY{n}{\PYZus{}prot}\PY{p}{)}
         \PY{n}{x}\PY{o}{.}\PY{n}{\PYZus{}\PYZus{}priv}
\end{Verbatim}

    \begin{Verbatim}[commandchars=\\\{\}]
I am public
I am public and my value can be changed
I am protected

    \end{Verbatim}

    \begin{Verbatim}[commandchars=\\\{\}]

        ---------------------------------------------------------------------------

        AttributeError                            Traceback (most recent call last)

        <ipython-input-15-7b9b44920b69> in <module>()
         11 print(x.pub + " and my value can be changed")
         12 print(x.\_prot)
    ---> 13 x.\_\_priv
    

        AttributeError: 'A' object has no attribute '\_\_priv'

    \end{Verbatim}

    The error message is very interesting. One might have expected a message
like "\_\_priv is private". We get the message "AttributeError: 'A'
object has no attribute '\_\_priv'" instead, which looks like a "lie".
There is such an attribute, but we are told that there isn't. This is
perfect information hiding. Telling a user that an attribute name is
private, means that we make some information visible, i.e. the existence
or non-existence of a private variable.

    \paragraph{Destructor:}\label{destructor}

What we said about constructors holds true for destructors as well.
There is no "real" destructor, but something similar, i.e. the method
\textbf{del}. It is called when the instance is about to be destroyed
and if there is no other reference to this instance. If a base class has
a \textbf{del}() method, the derived class's \textbf{del}() method, if
any, must explicitly call it to ensure proper deletion of the base class
part of the instance.

\textbf{\emph{The destructor was called after the program ended, not
when ft went out of scope inside make\_foo.}}

    \begin{Verbatim}[commandchars=\\\{\}]
{\color{incolor}In [{\color{incolor}38}]:} \PY{k}{class} \PY{n+nc}{FooType}\PY{p}{(}\PY{n+nb}{object}\PY{p}{)}\PY{p}{:}
             \PY{n+nb}{id} \PY{o}{=} \PY{l+m+mi}{0}
             \PY{k}{def} \PY{n+nf}{\PYZus{}\PYZus{}init\PYZus{}\PYZus{}}\PY{p}{(}\PY{n+nb+bp}{self}\PY{p}{,} \PY{n+nb}{id}\PY{p}{)}\PY{p}{:}
                 \PY{n+nb+bp}{self}\PY{o}{.}\PY{n}{id} \PY{o}{=} \PY{n+nb}{id}
                 \PY{n+nb}{print}\PY{p}{(}\PY{n+nb+bp}{self}\PY{o}{.}\PY{n}{id}\PY{p}{,} \PY{l+s+s1}{\PYZsq{}}\PY{l+s+s1}{born}\PY{l+s+s1}{\PYZsq{}}\PY{p}{)}
         
             \PY{k}{def} \PY{n+nf}{\PYZus{}\PYZus{}del\PYZus{}\PYZus{}}\PY{p}{(}\PY{n+nb+bp}{self}\PY{p}{)}\PY{p}{:}
                 \PY{n+nb}{print}\PY{p}{(}\PY{n+nb+bp}{self}\PY{o}{.}\PY{n}{id}\PY{p}{,} \PY{l+s+s1}{\PYZsq{}}\PY{l+s+s1}{died}\PY{l+s+s1}{\PYZsq{}}\PY{p}{)}
         
         
         \PY{n}{ft} \PY{o}{=} \PY{n}{FooType}\PY{p}{(}\PY{l+m+mi}{1}\PY{p}{)}
\end{Verbatim}

    \begin{Verbatim}[commandchars=\\\{\}]
1 born
1 died

    \end{Verbatim}

    \paragraph{Inheritance}\label{inheritance}

One of the major benefits of object oriented programming is reuse of
code and one of the ways this is achieved is through the inheritance
mechanism. Inheritance can be best imagined as implementing a type and
subtype relationship between classes.

Suppose you want to write a program which has to keep track of the
teachers and students in a college. They have some common
characteristics such as name, age and address. They also have specific
characteristics such as salary, courses and leaves for teachers and,
marks and fees for students.

You can create two independent classes for each type and process them
but adding a new common characteristic would mean adding to both of
these independent classes. This quickly becomes unwieldy.

    \begin{Verbatim}[commandchars=\\\{\}]
{\color{incolor}In [{\color{incolor}41}]:} \PY{k}{class} \PY{n+nc}{SchoolMember}\PY{p}{:}
             \PY{l+s+sd}{\PYZsq{}\PYZsq{}\PYZsq{}Represents any school member.\PYZsq{}\PYZsq{}\PYZsq{}}
             \PY{k}{def} \PY{n+nf}{\PYZus{}\PYZus{}init\PYZus{}\PYZus{}}\PY{p}{(}\PY{n+nb+bp}{self}\PY{p}{,} \PY{n}{name}\PY{p}{,} \PY{n}{age}\PY{p}{)}\PY{p}{:}
                 \PY{n+nb+bp}{self}\PY{o}{.}\PY{n}{name} \PY{o}{=} \PY{n}{name}
                 \PY{n+nb+bp}{self}\PY{o}{.}\PY{n}{age} \PY{o}{=} \PY{n}{age}
                 \PY{n+nb}{print}\PY{p}{(}\PY{l+s+s1}{\PYZsq{}}\PY{l+s+s1}{(Initialized SchoolMember: }\PY{l+s+si}{\PYZob{}\PYZcb{}}\PY{l+s+s1}{)}\PY{l+s+s1}{\PYZsq{}}\PY{o}{.}\PY{n}{format}\PY{p}{(}\PY{n+nb+bp}{self}\PY{o}{.}\PY{n}{name}\PY{p}{)}\PY{p}{)}
         
             \PY{k}{def} \PY{n+nf}{tell}\PY{p}{(}\PY{n+nb+bp}{self}\PY{p}{)}\PY{p}{:}
                 \PY{l+s+sd}{\PYZsq{}\PYZsq{}\PYZsq{}Tell my details.\PYZsq{}\PYZsq{}\PYZsq{}}
                 \PY{n+nb}{print}\PY{p}{(}\PY{l+s+s1}{\PYZsq{}}\PY{l+s+s1}{Name:}\PY{l+s+s1}{\PYZdq{}}\PY{l+s+si}{\PYZob{}\PYZcb{}}\PY{l+s+s1}{\PYZdq{}}\PY{l+s+s1}{ Age:}\PY{l+s+s1}{\PYZdq{}}\PY{l+s+si}{\PYZob{}\PYZcb{}}\PY{l+s+s1}{\PYZdq{}}\PY{l+s+s1}{\PYZsq{}}\PY{o}{.}\PY{n}{format}\PY{p}{(}\PY{n+nb+bp}{self}\PY{o}{.}\PY{n}{name}\PY{p}{,} \PY{n+nb+bp}{self}\PY{o}{.}\PY{n}{age}\PY{p}{)}\PY{p}{,} \PY{n}{end}\PY{o}{=}\PY{l+s+s2}{\PYZdq{}}\PY{l+s+s2}{ }\PY{l+s+s2}{\PYZdq{}}\PY{p}{)}
         
         
         \PY{k}{class} \PY{n+nc}{Teacher}\PY{p}{(}\PY{n}{SchoolMember}\PY{p}{)}\PY{p}{:}
             \PY{l+s+sd}{\PYZsq{}\PYZsq{}\PYZsq{}Represents a teacher.\PYZsq{}\PYZsq{}\PYZsq{}}
             \PY{k}{def} \PY{n+nf}{\PYZus{}\PYZus{}init\PYZus{}\PYZus{}}\PY{p}{(}\PY{n+nb+bp}{self}\PY{p}{,} \PY{n}{name}\PY{p}{,} \PY{n}{age}\PY{p}{,} \PY{n}{salary}\PY{p}{)}\PY{p}{:}
                 \PY{n}{SchoolMember}\PY{o}{.}\PY{n+nf+fm}{\PYZus{}\PYZus{}init\PYZus{}\PYZus{}}\PY{p}{(}\PY{n+nb+bp}{self}\PY{p}{,} \PY{n}{name}\PY{p}{,} \PY{n}{age}\PY{p}{)}
                 \PY{n+nb+bp}{self}\PY{o}{.}\PY{n}{salary} \PY{o}{=} \PY{n}{salary}
                 \PY{n+nb}{print}\PY{p}{(}\PY{l+s+s1}{\PYZsq{}}\PY{l+s+s1}{(Initialized Teacher: }\PY{l+s+si}{\PYZob{}\PYZcb{}}\PY{l+s+s1}{)}\PY{l+s+s1}{\PYZsq{}}\PY{o}{.}\PY{n}{format}\PY{p}{(}\PY{n+nb+bp}{self}\PY{o}{.}\PY{n}{name}\PY{p}{)}\PY{p}{)}
         
             \PY{k}{def} \PY{n+nf}{tell}\PY{p}{(}\PY{n+nb+bp}{self}\PY{p}{)}\PY{p}{:}
                 \PY{n}{SchoolMember}\PY{o}{.}\PY{n}{tell}\PY{p}{(}\PY{n+nb+bp}{self}\PY{p}{)}
                 \PY{n+nb}{print}\PY{p}{(}\PY{l+s+s1}{\PYZsq{}}\PY{l+s+s1}{Salary: }\PY{l+s+s1}{\PYZdq{}}\PY{l+s+si}{\PYZob{}:d\PYZcb{}}\PY{l+s+s1}{\PYZdq{}}\PY{l+s+s1}{\PYZsq{}}\PY{o}{.}\PY{n}{format}\PY{p}{(}\PY{n+nb+bp}{self}\PY{o}{.}\PY{n}{salary}\PY{p}{)}\PY{p}{)}
         
         
         \PY{k}{class} \PY{n+nc}{Student}\PY{p}{(}\PY{n}{SchoolMember}\PY{p}{)}\PY{p}{:}
             \PY{l+s+sd}{\PYZsq{}\PYZsq{}\PYZsq{}Represents a student.\PYZsq{}\PYZsq{}\PYZsq{}}
             \PY{k}{def} \PY{n+nf}{\PYZus{}\PYZus{}init\PYZus{}\PYZus{}}\PY{p}{(}\PY{n+nb+bp}{self}\PY{p}{,} \PY{n}{name}\PY{p}{,} \PY{n}{age}\PY{p}{,} \PY{n}{marks}\PY{p}{)}\PY{p}{:}
                 \PY{n}{SchoolMember}\PY{o}{.}\PY{n+nf+fm}{\PYZus{}\PYZus{}init\PYZus{}\PYZus{}}\PY{p}{(}\PY{n+nb+bp}{self}\PY{p}{,} \PY{n}{name}\PY{p}{,} \PY{n}{age}\PY{p}{)}
                 \PY{n+nb+bp}{self}\PY{o}{.}\PY{n}{marks} \PY{o}{=} \PY{n}{marks}
                 \PY{n+nb}{print}\PY{p}{(}\PY{l+s+s1}{\PYZsq{}}\PY{l+s+s1}{(Initialized Student: }\PY{l+s+si}{\PYZob{}\PYZcb{}}\PY{l+s+s1}{)}\PY{l+s+s1}{\PYZsq{}}\PY{o}{.}\PY{n}{format}\PY{p}{(}\PY{n+nb+bp}{self}\PY{o}{.}\PY{n}{name}\PY{p}{)}\PY{p}{)}
         
             \PY{k}{def} \PY{n+nf}{tell}\PY{p}{(}\PY{n+nb+bp}{self}\PY{p}{)}\PY{p}{:}
                 \PY{n}{SchoolMember}\PY{o}{.}\PY{n}{tell}\PY{p}{(}\PY{n+nb+bp}{self}\PY{p}{)}
                 \PY{n+nb}{print}\PY{p}{(}\PY{l+s+s1}{\PYZsq{}}\PY{l+s+s1}{Marks: }\PY{l+s+s1}{\PYZdq{}}\PY{l+s+si}{\PYZob{}:d\PYZcb{}}\PY{l+s+s1}{\PYZdq{}}\PY{l+s+s1}{\PYZsq{}}\PY{o}{.}\PY{n}{format}\PY{p}{(}\PY{n+nb+bp}{self}\PY{o}{.}\PY{n}{marks}\PY{p}{)}\PY{p}{)}
         
         \PY{n}{t} \PY{o}{=} \PY{n}{Teacher}\PY{p}{(}\PY{l+s+s1}{\PYZsq{}}\PY{l+s+s1}{Mr. Neel}\PY{l+s+s1}{\PYZsq{}}\PY{p}{,} \PY{l+m+mi}{24}\PY{p}{,} \PY{l+m+mi}{30000}\PY{p}{)}
         \PY{n}{s} \PY{o}{=} \PY{n}{Student}\PY{p}{(}\PY{l+s+s1}{\PYZsq{}}\PY{l+s+s1}{Mr. Water}\PY{l+s+s1}{\PYZsq{}}\PY{p}{,} \PY{l+m+mi}{20}\PY{p}{,} \PY{l+m+mi}{75}\PY{p}{)}
         
         \PY{c+c1}{\PYZsh{} prints a blank line}
         \PY{n+nb}{print}\PY{p}{(}\PY{p}{)}
         
         \PY{n}{members} \PY{o}{=} \PY{p}{[}\PY{n}{t}\PY{p}{,} \PY{n}{s}\PY{p}{]}
         \PY{k}{for} \PY{n}{member} \PY{o+ow}{in} \PY{n}{members}\PY{p}{:}
             \PY{c+c1}{\PYZsh{} Works for both Teachers and Students}
             \PY{n}{member}\PY{o}{.}\PY{n}{tell}\PY{p}{(}\PY{p}{)}
\end{Verbatim}

    \begin{Verbatim}[commandchars=\\\{\}]
(Initialized SchoolMember: Mr. Neel)
(Initialized Teacher: Mr. Neel)
(Initialized SchoolMember: Mr. Water)
(Initialized Student: Mr. Water)

Name:"Mr. Neel" Age:"24" Salary: "30000"
Name:"Mr. Water" Age:"20" Marks: "75"

    \end{Verbatim}

    \paragraph{Polymorphism}\label{polymorphism}

In a child class we can change how some methods work whilst keeping the
same name. We call this polymorphism or overriding and it is useful
because we do not want to keep introducing new method names for
functionality that is pretty similar in each class.

    \begin{Verbatim}[commandchars=\\\{\}]
{\color{incolor}In [{\color{incolor}43}]:} \PY{k}{class} \PY{n+nc}{A}\PY{p}{(}\PY{p}{)}\PY{p}{:}
          
             \PY{k}{def} \PY{n+nf}{\PYZus{}\PYZus{}init\PYZus{}\PYZus{}}\PY{p}{(}\PY{n+nb+bp}{self}\PY{p}{)}\PY{p}{:}
                 \PY{n+nb+bp}{self}\PY{o}{.}\PY{n}{\PYZus{}\PYZus{}x} \PY{o}{=} \PY{l+m+mi}{1}
          
             \PY{k}{def} \PY{n+nf}{message}\PY{p}{(}\PY{n+nb+bp}{self}\PY{p}{)}\PY{p}{:}
                 \PY{n+nb}{print}\PY{p}{(}\PY{l+s+s2}{\PYZdq{}}\PY{l+s+s2}{message from A}\PY{l+s+s2}{\PYZdq{}}\PY{p}{)}
          
          
         \PY{k}{class} \PY{n+nc}{B}\PY{p}{(}\PY{n}{A}\PY{p}{)}\PY{p}{:}
          
             \PY{k}{def} \PY{n+nf}{\PYZus{}\PYZus{}init\PYZus{}\PYZus{}}\PY{p}{(}\PY{n+nb+bp}{self}\PY{p}{)}\PY{p}{:}
                 \PY{n+nb+bp}{self}\PY{o}{.}\PY{n}{\PYZus{}\PYZus{}y} \PY{o}{=} \PY{l+m+mi}{1}
          
             \PY{k}{def} \PY{n+nf}{message}\PY{p}{(}\PY{n+nb+bp}{self}\PY{p}{)}\PY{p}{:}
                 \PY{n+nb}{print}\PY{p}{(}\PY{l+s+s2}{\PYZdq{}}\PY{l+s+s2}{message from B}\PY{l+s+s2}{\PYZdq{}}\PY{p}{)}
         
         \PY{n}{try\PYZus{}1} \PY{o}{=} \PY{n}{A}\PY{p}{(}\PY{p}{)}
         \PY{n}{try\PYZus{}1}\PY{o}{.}\PY{n}{message}\PY{p}{(}\PY{p}{)}
         
         \PY{n}{try\PYZus{}2} \PY{o}{=} \PY{n}{B}\PY{p}{(}\PY{p}{)}
         \PY{n}{try\PYZus{}2}\PY{o}{.}\PY{n}{message}\PY{p}{(}\PY{p}{)}
\end{Verbatim}

    \begin{Verbatim}[commandchars=\\\{\}]
message from A
message from B

    \end{Verbatim}

    \textbf{This is the end of the OOP concept. Hope you like it and not
bored!}

You can download all material from github account or website.

\begin{enumerate}
\def\labelenumi{\arabic{enumi})}
\item
  Website: https://neelshah18.github.io/
\item
  GitHub: https://github.com/NeelShah18/OOP-with-python
\end{enumerate}

    \paragraph{References:}\label{references}

\begin{enumerate}
\def\labelenumi{\arabic{enumi})}
\item
  http://www.python-course.eu/object\_oriented\_programming.php
\item
  http://thepythonguru.com/python-inheritance-and-polymorphism/
\item
  https://pythonschool.net/oop/inheritance-and-polymorphism/
\end{enumerate}


    % Add a bibliography block to the postdoc
    
    
    
    \end{document}
